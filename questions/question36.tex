\section{Минимальное число инверторов, достаточное для реализации системы функций \{$\overline{x}$, $\overline{y}$, $\overline{z}$\} в базисе \{$x\vee y$, $x \& y$, $\overline{x}$\}.}

\begin{theorem}
    Для реализации истемы функций \{$\overline{x}$, $\overline{y}$, $\overline{z}$\} в базисе $B_0$ нужно хотя бы 2 инвертора.
\end{theorem}

\begin{proof}
    Очевидно, необходим хотя бы инвертор, поскольку $x\vee y$, $xy$ --- монотоннные функции, а значит, любая схема из этих функций реализует монотонную функцию.

    Покажем, что одного инвертора недостаточно. Предположим, что в схеме был ровно один инвертор. Тогда, при подаче $(x, y) = (0, 1)$ схема выдавала $(1, 0)$, а инвертор мог выдавать 2 значения:
    \begin{enumerate}
        \item Если инвертор выдавал 0, то подадим $(x,y) = (1, 1)$. Тогда всё, что было до инвертора неуменьшится, а значит, инвертор так и будет выдавать 0, и то, что вне инвертора тоже не уменьшится (в силу монотонности). Значения, которые выдает схема должны были не уменьшится, но в этом случае схема должна выдавать $(0, 0)\leqslant (1, 0)$, противоречие.
        \item Если инвертор выдавал 1, то подадим $(x, y) = (0, 0)$. Тогда всё, что было до инвертора не увеличится, а значит, инвертор так и будет выдавать 1, и то, что вне инвертора тоже не увеличится (в силу монотонности). Значения, которые выдает схема должны были не увеличится, но в этом случае схема должна выдавать $(1, 1)\geqslant(1, 0)$, противоречие.
    \end{enumerate}

    Теперь приведём пример схемы, состоящий из двух инверторов. Снача построим все симметрические функции от трёх переменных. Обозначим $s^{i_1, \ldots, i_k}$ --- симметрическая функция, которая принимает 1 на наборах из $i_1, \ldots, i_k$ единиц, и 0 иначе. Построим $s^0, s^1, s^2, s^3$ --- функции от 3-x переменных:
    \begin{gather*}
        \textcolor{red}{s^3 = xyz},\quad s^{23} = m(x, y, z) = xy\vee xz\vee yz, \quad s^{123} = x\vee y \vee z,\\
        s^{01} = \overline{s^{23}}, \quad \textcolor{red}{s^1 = s^{01} \cdot  s^{123}}, \quad s^{13} = s^1 \vee s^3, \quad s^{02} = \overline{s^{13}},\\
        \textcolor{red}{s^0 = s^{01} \cdot s^{02}}, \quad \textcolor{red}{s^2 = s^{02}\cdot s^{23}}.
    \end{gather*}
Далее получим все функции вида $x^{\sigma_1}y^{\sigma_2}z^{\sigma_3}$:

    \begin{enumerate}[nolistsep]
        \item $xyz = s^3$.
        \item $xy\overline{z} = xy \cdot s^2$, поскольку $s^2 = \overline{x}yz \vee x\overline{y}z \vee xy\overline{z}$. Остальные конъюнкции с одним отрицанием аналогично.
        \item $x\overline{y}\overline{z} = x \cdot s^1$, поскольку $s^1 = \overline{x}\overline{y}z \vee \overline{x}y\overline{z} \vee x\overline{y}\overline{z}$. Остальные конъюнкции с двумя отрицаниями аналогично.
        \item $\overline{x}\overline{y}\overline{z} = s^0$.
    \end{enumerate}
    Таким образом, можно построить схему для любой системы функций от трёх переменных (СДНФ), в том числе \{$\overline{x}$, $\overline{y}$, $\overline{z}$\}.
\end{proof}