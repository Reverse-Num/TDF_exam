\section{Деревья, характеристические свойства деревьев.}

\begin{definition}
    Неориентированный связный граф без циклов называется \textit{деревом}.
\end{definition}

\begin{theorem}
    Пусть $G$ --- конечный обыкновенный граф. Тогда следующие высказывания равносильны:
    \begin{enumerate}
        \item Граф $G$ --- дерево.
        \item В графе $G$ любые две вершины соединены единственной цепью.
        \item Граф $G$ связен и число ребер на единицу меньше числа вершин.
        \item Граф $G$ связен, но при удалении любого ребра перестает быть связным.
        \item Граф $G$ не содержит циклов, но при добавлении любого ребра образуется цикл.
    \end{enumerate}
\end{theorem}

\begin{exercise}
    Докажите теорему 1.
\end{exercise}