\section{Предполные классы. Теорема о предполных классах в $P_2$.}

\begin{lemma}
    Для любых двух классов из множества ${T_0,T_1, L, S, M}$ найдется функция, лежащая в одном и не лежащая в другом.
\end{lemma}

\begin{proof}
    Рассмотрим таблицу:
    \[
        \begin{array}{r | c c c c c}
            & T_0 & T_1 & S & M & L\\
            \hline
            0 & + & - & - & + & +\\
            1 & - & + & - & + & +\\
            \overline{x} & - & - & + & - & +\\
            x \cdot y & + & + & - & + & -\\
            x \oplus y & + & - & - & - & +\\
            x \oplus y \oplus z & + & + & + & - & +\\
            m(x, y, z) & + & + & + & + & -\\
        \end{array}
    \]
    $m(x, y, z)$ --- \textit{функция голосования} (см. билет 2). В приведённой таблице для каждой упорядоченной пары классов существует функция, содержащаяся в первом, но не содержащаяся во втором.
\end{proof}

\begin{definition}
    Класс функций $A \in P_2$ называется \textit{предполным}, если
    \begin{enumerate}[nolistsep]
        \item система $A$ не полная;
        \item $\forall f \notin A$ система $A \cup \{f\}$ --- полная.
    \end{enumerate}
\end{definition}

\begin{remark}
    Любой предполный класс замкнут.
\end{remark}

\begin{proof}
    Предположим противное. Рассмотрим $f \in [K] \setminus K$. $[K \cup \{f\}] = P_2$. С другой стороны, $K \cup \{f\} \subseteq [K]$, а значит, $[K \cup \{f\}] \subseteq [[K]] = [K] \ne P_2$. Противоречие.
\end{proof}

\begin{theorem}
    В $P_2$ ровно 5 предполных классов: $T_0$, $T_1$, $L$, $S$, $M$.
\end{theorem}

\begin{proof}
    Действительно, если класс $K$ предполон, то он, согласно следствию 1 из билета 11, должен содержаться в одном из классов $T_0$, $T_1$, $L$, $S$, $M$. Пусть он содержится в классе $Q$. $K$ не может быть собственным подмножеством $Q$, поскольку можно взять $f \in Q \setminus K$, и тогда $[K\cup \{f\}] \subseteq Q$. Противоречие, предполными классами могут быть только $T_0$, $T_1$, $L$, $S$, $M$.

    Теперь докажем, что все они --- предполные классы. Рассмотрим произвольный класс $Q$ из указанных пяти классов. Возьмём произвольную функцию алгебры логики $f$, не принадлежащую $Q$. Тогда $[Q \cup\{f\}] = P_2$, поскольку ни один класс полностью не содержится в другом, и $f \notin Q$.
\end{proof}
