\section{Реализация симметрических булевых функций.}

\begin{definition}
    Обозначим через $\Sigma_n$ булев оператор суммирования $n$-разрядных чисел, т.\,е. булеву $(2n, n + 1)$-функцию, которая по двум $n$-разрядным двоичным числам вычисляет $(n + 1)$-разрядное двоичное представление их суммы.
\end{definition}

\begin{definition}
     Обозначим через $N_n$ --- булев оператор подсчета числа единиц в наборе длины $n$, т.\,е. булеву $(n, \ceil{\log(n + 1)})$ функцию, которая по $n$-разрядному двоичному набору вычисляет $\ceil{\log(n+1)}$-разрядное двоичное представление количества единиц в этом наборе.
\end{definition}

\begin{lemma}
    Для любого конечного полного базиса $B$ при $n \to \infty$ справедливо равенство 
    \[
        L_B(\Sigma_n) = O(n).
    \]
\end{lemma}

\begin{proof}
    Пусть базис $A$ содержит функции $x\& y$, $x\oplus y$, $x \oplus y \oplus z$ и $m(x, y, z)$. Для построения нашей схемы построим две подсхемы $\Sigma_0$ --- на вход подаётся два последних бита, а на выход сумма разрядов по модулю 2 и перенос <<десятка>> (на деле двойки), $\Sigma$ --- на вход подаётся два бита и перенос <<десятка>>, а на выход сумма по модулю 2 и перенос <<десятка>>.
    \begin{figure}[h]
        \centering
        \begin{minipage}[b]{0.48\textwidth}
            \begin{center}
                \begin{tikzpicture}[
                    circuit/.style={rectangle, draw=black, minimum size=6mm}
                    ]
                    %Nodes
                    \node[circuit,scale=2](sum) at (0, 0) {$\oplus$};
                    \node[circuit,scale=2](and) at (2, 0) {$\&$};
                    \node (a_1) at (0, 2) {$a_1$};
                    \node (b_1) at (2, 2) {$b_1$};
                    \node (c_1) at (0, -1.6) {$c_1$};
                    \node (z_2) at (2, -1.6) {$z_2$};
                    %Lines
                    \draw[arrows = {Circle[length=3pt, width=3pt, open]}-] (a_1) -- (0, 1) -- (-0.3, 1) -- (-0.3, 0.6);
                    \draw[arrows = {Circle[length=3pt, width=3pt, open]}-] (b_1) -- (2, 0.8) -- (0.3, 0.8) -- (0.3, 0.6);
                    \draw (0, 1) -- (1.7, 1) -- (1.7, 0.6);
                    \draw (2, 0.8) -- (2.3, 0.8) -- (2.3, 0.6);
                    \draw[arrows = {-Circle[length=3pt, width=3pt]}] (0, -0.6) -- (c_1);
                    \draw[arrows = {-Circle[length=3pt, width=3pt]}] (2, -0.6) -- (z_2);
                    \draw (-1, -1) rectangle (3, 1.4);
                \end{tikzpicture}
            \end{center}
            \caption{Реализация $\Sigma_0$}
        \end{minipage}
        \hfill
        \begin{minipage}[b]{0.48\textwidth}
            \begin{center}
                \begin{tikzpicture}[
                    circuit/.style={rectangle, draw=black, minimum size=6mm}
                    ]
                    %Nodes
                    \node[circuit,scale=2](sum) at (0, 0) {$\oplus$};
                    \node[circuit,scale=2](m) at (2, 0) {$m$};
                    \node (a) at (0, 2) {$a_i$};
                    \node (b) at (1, 2) {$b_i$};
                    \node (z_i) at (2, 2) {$z_i$};
                    \node (c) at (0, -1.6) {$c_i$};
                    \node (z) at (2, -1.6) {$z_{i+1}$};
                    %Lines
                    \draw[arrows = {Circle[length=3pt, width=3pt, open]}-] (a) -- (0, 1.2) -- (-0.3, 1.2) -- (-0.3, 0.6);
                    \draw[arrows = {Circle[length=3pt, width=3pt, open]}-] (b) -- (1, 1) -- (0, 1) -- (0, 0.6);
                    \draw[arrows = {Circle[length=3pt, width=3pt, open]}-] (z_i) -- (2, 0.8) -- (0.3, 0.8) -- (0.3, 0.6);
                    \draw (0, 1.2) -- (1.7, 1.2) -- (1.7, 0.6);
                    \draw (1, 1) -- (2.3, 1) -- (2.3, 0.6);
                    \draw (2, 0.8) -- (2, 0.6);
                    \draw (0, -0.6)[arrows = {-Circle[length=3pt, width=3pt]}] -- (c);
                    \draw (2, -0.6)[arrows = {-Circle[length=3pt, width=3pt]}] -- (z);
                    \draw (-1, -1) rectangle (3, 1.4);
                \end{tikzpicture}
            \end{center}
        \caption{Реализация $\Sigma$}
        \end{minipage}
    \end{figure}
    
    Построим схему $S$, которая по двум группам входов --- $(a_1, \ldots, a_n)$ и
    $(b_1, \ldots, b_n)$, на которые подаются двоичные $n$-разрядные числа (младшие разряды $a_1$ и $b_1$), вычисляет набор $(c_1, \ldots, c_{n+1})$, представляющий двоичную запись их суммы. Тогда, обозначив через $z_i, i = 2, \ldots, n + 1$, значение переноса в $i$-й разряд, получаем:
    \begin{gather*}
        c_1 = a_1 \oplus b_1, z_2 = a_1 \& b_1, \\
        c_i = a_i \oplus b_i \oplus z_i, z_{i+1} = m(a_i, b_i, z_i),\\
        c_{n+1} = z_{n+1},
    \end{gather*}
    или же в виде схемы:
    \begin{figure}[h]
        \centering
        \begin{center}
            % Это магия, хз почему это работает
            \begin{tikzpicture}[
                circuit/.style={rectangle, draw=black, minimum size=7mm}
                ]
                %Nodes
                \node[circuit,scale=1.71428](sigma1) at (0, 0) {$\Sigma$};
                \node[scale=1.71428] at (2, 0) {$\ldots$};
                \node[circuit,scale=1.71428](sigma2) at (4, 0) {$\Sigma$};
                \node[circuit,scale=1.71428](sigma_0) at (6, 0) {$\Sigma_0$};
                \node (a_n) at (-0.3, 2) {$a_n$};
                \node (dots1) at (0.56666, 2) {$\ldots$};
                \node (a_2) at (1.43333, 2) {$a_2$};
                \node (a_1) at (2.3, 2) {$a_1$};
                \node (b_n) at (3.7, 2) {$b_n$};
                \node (dots2) at (4.56666, 2) {$\ldots$};
                \node (b_2) at (5.43333, 2) {$b_2$};
                \node (b_1) at (6.3, 2) {$b_1$};
                \node (c_1) at (6.3, -1.6) {$c_1$};
                \node (z_2) at (5.7, -1.6) {$z_2$};
                \node (c_2) at (4.3, -1.6) {$c_2$};
                \node (z_3) at (3.7, -1.6) {$z_3$};
                \node (c_n) at (0.3, -1.6) {$c_n$};
                \node (z_n+1) at (-0.3, -1.6) {$z_{n+1}$};
                
                %Lines
                \draw[arrows = {-Circle[length=3pt, width=3pt]}] (6.3, -0.6) -- ++ (c_1);
                \draw (5.7, -0.6) -- (z_2.north) -- ++ (-0.7, 0) -- (5, 0.8) -- (4.3, 0.8) -- (4.3, 0.6);
                \draw[arrows = {-Circle[length=3pt, width=3pt]}] (4.3, -0.6) -- ++ (c_2);
                \draw (3.7, -0.6) -- (z_3.north) -- ++ (-0.7, 0) -- (3, 0.8) -- (2.3, 0.8);
                \draw[arrows = {-Circle[length=3pt, width=3pt]}] (0.3, -0.6) -- ++ (c_n);
                \draw (z_3.north) ++ (-2, 0) -- ++ (-0.7, 0) -- (1, 0.8) -- (0.3, 0.8) -- (0.3, 0.6);
                \draw[arrows = {-Circle[length=3pt, width=3pt]}] (-0.3, -0.6) -- ++ (z_n+1);
                \draw[arrows = {Circle[length=3pt, width=3pt, open]}-] (a_n) -- (-0.3, 0.6);
                \draw[arrows = {Circle[length=3pt, width=3pt, open]}-] (b_n) -- ++ (0, -0.6) -- (0, 1.4) -- (0, 0.6);
                \draw[arrows = {Circle[length=3pt, width=3pt, open]}-] (a_2) -- ++ (0, -1) -- (3.7, 1) -- (3.7, 0.6);
                \draw[arrows = {Circle[length=3pt, width=3pt, open]}-] (b_2) -- ++ (0, -0.5) -- (4, 1.5) -- (4, 0.6);
                \draw[arrows = {Circle[length=3pt, width=3pt, open]}-] (a_1) -- ++ (0, -0.8) -- (5.7, 1.2) -- (5.7, 0.6);
                \draw[arrows = {Circle[length=3pt, width=3pt, open]}-] (b_1) -- ++ (0, -1.4);
            \end{tikzpicture}
        \end{center}
        \caption{Реализация $\Sigma_n$}
    \end{figure}
    
    На каждый сумматор уходит 2 функции, а всего их $n$. Следовательно, пользуясь леммой, имеем
    \[
        L_A (\Sigma_n) \leqslant 2n \Rightarrow L_B(\Sigma_n) = O(n).
    \]
\end{proof}

\begin{lemma}
    Для любого конечного полного базиса $B$ при $n \to \infty$ справедливо равенство 
    \[
        L_B(N_n) = O(n).
    \]
\end{lemma}

\begin{proof}
    Очевидно, что результат применения оператора $N_n$ равен двоичной записи суммы $n$ подаваемых на входы оператора одноразрядных двоичных чисел. Опишем способ вычисления этой суммы схемой линейной сложности. Сначала будем считать, что $n = 2^k$ для некоторого $k$. Построим схему $S$, имеющую $k$ ярусов. Ярус с номером $t$, $t = 1,\ldots, k$, будет состоять из $2^{k-t}$ подсхем, каждая из которых реализует оператор $\Sigma_t$ и, следовательно, имеет две группы по $t$ входов, а также $t+1$ выходов. Таким образом, считая в силу леммы 1, что $L_B(\Sigma_t) \leqslant ct$, в случае, когда $n = 2^k$, имеем:
    \[
        L_B(N_n) \leqslant L_B(S) \leqslant \sum_{t=1}^k 2^{k-t}ct = c2^k \sum_{t=1}^k \frac{t}{2^t} < 2c 2^k = 2cn.
    \]
    Переходя к общему случаю, полагаем $n' = 2^{\ceil{\log n}}$. Очевидно, что $n \leqslant n' < 2n$. Схему, реализующую оператор $N_n$, можно получить из схемы $S′$, реализующей оператор $N_{n′}$, подав на $n' - n$ входов схемы $S'$ константу 0. Поэтому 
    \[
    L_B(N_n) \leqslant L_B(0) + L_B(N_{n'}) \leqslant L_B(0) + 2cn'
    \leqslant L_B(0) + 4cn = O(n).
    \]
\end{proof}

\begin{definition}
    Функция $f(x_1, \ldots, x_n)$ называется \textit{симметрической}, если для любой перестановки $\sigma$ из симметрической группы $S_n$ выполняется равенство $f(x_{\sigma(1)}, \ldots, x_{\sigma(n)}) = f(x_1, \ldots, x_n)$.
\end{definition}

\begin{theorem}
    Для любого полного конечного базиса $B$ существуют $c_1, c_2 > 0$ такие, что
    для любой симметрической булевой функции $f(x_1, \ldots, x_n)$, отличной от константы, выполняются неравенства
    \[
        c_1n\leqslant L_B(f(x_1, \ldots, x_n)) \leqslant c_2n.
    \]
\end{theorem}

\begin{proof}
    Нижняя оценка следует из неравенства $L_B(f) \geqslant \frac{n-1}{r(B) - 1}$ (см. билет 26).

    Переходя к доказательству верхней оценки отметим, что произвольная симметрическая функция $f$ от $n$ переменных может быть задана двоичной последовательностью $\widetilde{\pi}(f) = (\pi_0(f), \ldots, \pi_n(f))$, где $\pi_k(f)$ значение функции $f$ на наборах, состоящих из $k$ единиц и $n - k$ нулей. На этом и основан метод построения схемы $S$, вычисляющей функцию $f(x_1, \ldots, x_n)$.
    
    Схема $S$ состоит из подсхем $S_1$ и $S_2$. Подсхема $S_1$ реализует оператор $N_n$, на выходах подсхемы $S_1$ вычисляется двоичная запись длины $\ceil{\log(n + 1)}$ числа единиц во входном наборе. Подсхема $S_2$ по двоичной записи числа единиц во входном наборе вычисляет значение функции $f$ на этом наборе. В силу леммы 2 и верхней оценки функции Шенона, имеем
    \[
        L_B(f) \leqslant L_B(N_n) + L_B(\ceil{\log(n+1)}) = O(n) + O\left(\frac{n}{\log n}\right) = O(n).
    \]
\end{proof}