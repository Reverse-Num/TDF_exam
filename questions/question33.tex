\section{Порядок роста функции Шеннона в произвольном полном конечном базисе.}

\begin{theorem}
    Пусть $B$ --- конечный полный базис, тогда существуют $a, b > 0$ такие, что при $n\to \infty$ выполняется асимптотическое неравенство:
    \[
        a\frac{2^n}{n} \lesssim L_B(n) \lesssim b\frac{2^n}{n}.
    \]
\end{theorem}

\begin{proof}
    Прямое следствие леммы 1 из билета 30 и верхней и нижней оценок функции Шеннона.
\end{proof}