\section{Схемы из функциональных элементов в базисе \{$x\vee y$, $x \& y$, $\overline{x}$\}. Определение функций, реализуемых в вершинах схемы. Независимость функций, реализуемых в вершинах схемы, от выбора монотонной нумерации вершин. Формулы как схемы. Схемы вычислений.}

Пусть есть:
\begin{enumerate}[nolistsep]
    \item Множество <<исходных данных>> $X$ (как правило, это переменные и, быть может, константы; в нашем случае $X = \{x_1, \ldots , x_n\}$);
    \item Множество <<базисных операций>> $B$ (в нашем случае $B_0 = \{x \vee
y, x\&y, \overline{x}\}$).
\end{enumerate}

\begin{definition}
    \textit{Схемой из функциональных элементов} в базисе $B_0$ называется ориентированный граф без ориентированных циклов, в котором входные степени вершин могут быть равны только 0, 1 или 2, при этом если входная степень вершины равна 0, то вершине приписывается символ переменной из множества $X$ (такие вершины называются входами), если входная степень вершины равна 1, то вершине приписывается функциональный символ, соответствующий операции отрицания, а если входная степень вершины равна 2, то вершине приписывается функциональный символ, соответствующий либо двухместной конъюнкции, либо двухместной дизъюнкции. Вершины с ненулевой входной степенью (т.\,е. вершины, которым приписаны символы операций), будем называть \textit{функциональными элементами}. Кроме того, одна или несколько вершин помечены дополнительно <<звездочкой>> --- эти вершины называются выходами (с них считывается информация).
\end{definition}

Зафиксируем какую-либо правильную нумерацию вершин схемы. Далее в порядке увеличения номера естественным образом приписываем вершине вычисляемую функцию. Тем самым каждой вершине будет приписана своя функция.

\begin{definition}
    Будем говорить, что СФЭ \textit{реализует} (\textit{вычисляет}) булеву функцию $f(x_1, \ldots, x_n)$ (систему функций $f_1(x_1, \ldots, x_n), \ldots, f_m(x_1, \ldots, x_n))$, если выходу (выходам) приписана эта функция (эта система функций). Удобно считать, что выходы СФЭ пронумерованы (упорядочены), и тем самым СФЭ вычисляет булеву $(n, m)$-функцию --- вектор (набор) из $m$ булевых функций от $n$ переменных.

\end{definition}

Отметим, что при всём формальном различии в определениях СФЭ и формулы, любую формулу можно интерпретировать как СФЭ.

\begin{definition}
    \textit{Сложностью} схемы $S$ называется число функциональных элементов схемы $S$.
\end{definition}

\begin{definition}
    \textit{Схема вычислений} (над $X$) в базисе $B$ --- это последовательность равенств
    \begin{gather*}
        z_1 = \varphi_1(y_{11}, \ldots, y_{1r_1});\\
        \ldots \ldots \ldots\\
        z_i = \varphi_i(y_{i1}, \ldots, y_{ir_i});\\
        \ldots \ldots \ldots\\
        z_l = \varphi_l(y_{l1}, \ldots, y_{lr_l}),
    \end{gather*}
    где каждая переменная $y_{ij}$ ($i = 1, \ldots, l; j = 1, \ldots, r_i$) --- это либо одна из входных независимых переменных из множества $X$, либо одна из внутренних переменных $z_1, \ldots, z_{i-1}$, вычисленных на предыдущих шагах; $\varphi_1, \ldots, \varphi_l \in B$. Кроме того, одна или несколько внутренних переменных из множества $z_1, \ldots, z_l$ дополнительно помечены <<звёздочкой>> --- эти переменные называются выходными (с них считывается информация).
\end{definition}

\begin{theorem}
    Функция, которую вычисляет СФЭ, не зависит от выбора монотонной нумерации.
\end{theorem}

\begin{proof}
    Рассмотрим две различные монотонные нумерации $Num_1$ и $Num_2$. Пусть есть номер $i$ --- наименьший номер вершины в $Num_1$, что в этой вершине вычисляемая функция в $Num_1$ отличается от вычисляемой функции в $Num_2$. Т.\,к. нумерация монотонная, то все входы у функционального элемента в обоих нумерациях будут совпадать, а во-вторых функциональный символ этой вершины совпадает, а значит, и функция реализумая в этой вершине будет совпадать в обоих нумерациях. Противоречие, а значит, функция, которую вычисляет СФЭ не зависит от выбора монотонной нумерации.
\end{proof}