\section{Асимптотически оптимальная по сложности реализация системы всех элементарных конъюнкций длины $n$.}

Обозначим через $K_n$ множество всех $2^n$ элементарных конъюнкций от $n$ переменных:
\[
    K_n(x_1, \ldots, x_n) = \{x_1^{\sigma_1}\ldots x_n^{\sigma_n} \vert\, (\sigma_1, \ldots, \sigma_n) \in B_n\}.
\]

\begin{theorem}
    При $n \to \infty$ справедливо асимптотическое соотношение:
    \[
        L(K_n) \sim 2^n.
    \]
\end{theorem}

\begin{proof}
    Построим схему, реализующую систему функций $K_n(x_1, \ldots, x_n)$, как показано на рисунке: схема состоит из трёх блоков (подсхем).
    %Тут очень плохо сделано, но уже лень переделывать.
    \begin{center}
        \begin{tikzpicture}[
            smallrectangle/.style={rectangle, draw=red!60, fill=red!5, very thick, minimum size=5mm, minimum width=4cm, minimum height=0.8cm}, bigrectangle/.style={rectangle, draw=blue!60, fill=blue!5, very thick, minimum size=5mm, minimum width=9.2cm, minimum height=0.8cm}
            ]
            %Nodes
            \node[bigrectangle](big_rectangle) at (4.6,-1.6) {<<$\&$>>};
            \node[smallrectangle](left_rectangle) at (2,0)  {$K_{\floor{n/2}}(x_1, \ldots, x_{\floor{n/2}})$};
            \node[smallrectangle](right_rectangle) at (7,0) {$K_{\ceil{n/2}}(x_{\floor{n/2}+1}, \ldots, x_n)$};
            \node (x1) at (0.5, 1.2) {$x_1$};
            \node (...1) at (2, 1.2) {$\ldots \quad \ldots$};
            \node (xn/2-1) at (3.5, 1.2) {$x_{\floor{n/2}}$};
            \node (xn/2+1) at (5.5, 1.2) {$x_{\floor{n/2}+1}$};
            \node (...2) at (7.1, 1.2) {$\ldots \quad \ldots$};
            \node (xn) at (8.7, 1.2) {$x_n$};
            \node (f_1) at (0.5, -2.8) {$f_1$};
            \node (f_2^n) at (8.7, -2.8) {$f_{2^n}$};
            \node (bigldots) at (4.6, -2.8) {$\ldots \quad \ldots \quad \ldots \quad \ldots \quad \ldots \quad \ldots \quad \ldots \quad \ldots$};
            
            %Lines
            \draw[arrows = {-Latex[width=3pt, length=3pt]}] (left_rectangle.south) -- (2, -1.2);
            \draw[arrows = {-Latex[width=3pt, length=3pt]}] (right_rectangle.south) -- (7, -1.2);
            \draw[arrows = {-Circle[length=3pt, width=3pt]}] (0.5, -2) -- (f_1);
            \draw[arrows = {-Circle[length=3pt, width=3pt]}] (8.7, -2) -- (f_2^n);
            \draw[arrows = {Circle[length=3pt, width=3pt, open]}-] (x1) -- (0.5, 0.4);
            \draw[arrows = {Circle[length=3pt, width=3pt, open]}-] (xn/2-1) -- (3.5, 0.4);
            \draw[arrows = {Circle[length=3pt, width=3pt, open]}-] (xn/2+1) -- (5.5, 0.4);
            \draw[arrows = {Circle[length=3pt, width=3pt, open]}-] (xn) -- (8.7, 0.4);
            % \draw (0, 0) rectangle (4, 1);
            % \draw (2, 0.5) node {$K_{\floor{n/2}}(x_1, \ldots, x_{\floor{n/2}})$};
            % \draw (8, 0) rectangle (12, 1);
        \end{tikzpicture}
    \end{center}
    Первая подсхема по переменным $x_1, \ldots, x_{\floor{n/2}}$ реализует систему конъюнкций $K_{\floor{n/2}}(x_1, \ldots, x_{\floor{n/2}})$, вторая подсхема по переменным $x_{\floor{n/2}+1}, \ldots, x_n$ реализует $K_{\ceil{n/2}}(x_{\floor{n/2}+1}, \ldots, x_n)$, а третья --- каждую элементарную конъюнкцию из множества $K_n(x_1, \ldots, x_n)$ <<собирает>> (с помощью одной операции конъюнкции) из двух ее <<половинок>>, реализованных на некоторых выходах первой и второй подсхемы соответсвенно. Все конъюнкции длины $k$ можно тривиальным образом релизовать за $k+(k-1)2^k$ операций ($k$ отрицаний, $k-1$ конъюнкций $2^k$ элементарных конъюнкицй). Имеем:
    \begin{multline*}
        L(K_n) \leqslant L(K_{\floor{n/2}}) + L(K_{\ceil{n/2}}) + 2^n \leqslant \\
        \leqslant \floor{\dfrac{n}{2}} + \left(\floor{\dfrac{n}{2}}-1\right)2^{\floor{n/2}} + \ceil{\dfrac{n}{2}} + \left(\ceil{\dfrac{n}{2}}-1\right)2^{\ceil{n/2}} + 2^n \leqslant 2^n + O(n\sqrt{2^n}).
    \end{multline*}
\end{proof}


