\section{Разложение булевой функции по одной и нескольким переменным. Совершенная дизъюнктивная нормальная форма (СДНФ). Полнота системы \{$x\vee y$, $x \& y$, $\overline{x}$\}. Совершенная конъюнктивная нормальная форма (СКНФ).}

\begin{definition}
    Пусть $x$ --- переменная, $\sigma \in E$. Тогда 
    $
    x^\sigma \vcentcolon =
    \begin{cases}
        x,&\text{если $\sigma = 1$},\\
        \overline{x},&\text{если $\sigma = 0$}.
    \end{cases}
    $
\end{definition}

\begin{remark}
    $x^\sigma = 1 \Leftrightarrow x = \sigma$.
\end{remark}

\begin{definition}
    $\sum\limits_{i = 1}^na_i \vcentcolon = a_1 \oplus a_2 \oplus \ldots \oplus a_n$, $\prod\limits_{i = 1}^na_i \vcentcolon = a_1 \cdot a_2 \cdot \ldots \cdot a_n$, $\bigvee\limits_{i = 1}^na_i \vcentcolon = a_1 \vee a_2 \vee \ldots \vee a_n$.
\end{definition}

\begin{theorem}
    $\forall f \in P_2(n)$, $\forall m = 1, \ldots, n$
    \[
        f(x_1, \ldots, x_n) = \bigvee\limits_{(\sigma_1, \ldots, \sigma_m) \in E^m}\prod\limits_{i = 1}^mx_i^{\sigma_i} \cdot f(\sigma_1, \ldots, \sigma_m, x_{m + 1}, \ldots, x_n).
    \]
\end{theorem}

\begin{proof}
    Рассмотрим произвольный двоичный набор $(\alpha_1, \ldots, \alpha_m)$. Если $(\sigma_1, \ldots, \sigma_m) \ne (\alpha_1, \ldots, \alpha_m)$, то найдётся $i \in \{1, \ldots, m\}$, для которого $\sigma_i \ne \alpha_i$. Тогда $\alpha_i^{\sigma_i} = 0$. Единственным членом дизъюнкции, влияющим на её значение является $(\sigma_1, \ldots, \sigma_m) = (\alpha_1, \ldots, \alpha_m)$. Он равен $\alpha_1^{\sigma_1}\ldots\alpha_m^{\sigma_m}f(\alpha_1, \ldots, \alpha_m, x_{m + 1}, \ldots, x_n) = f(\alpha_1, \ldots, \alpha_m, x_{m+1}, \ldots, x_n)$.
\end{proof}

\begin{corollary}
    При $m = 1$ получаем, так называемое, \textit{разложение функции $f$ по переменной $x_n$}:
    \[f(x_1, \ldots, x_n) = x_n \cdot f(x_1, \ldots, x_{n - 1}, 1) \vee \overline{x}_n \cdot f(x_1, \ldots, x_{n - 1}, 0).\]
\end{corollary}

\begin{corollary}
    При $m = n$ получаем \textit{совершенную дизъюнктивную нормальную форму}:
    \[f(x_1, \ldots, x_n) = \bigvee\limits_{(\sigma_1, \ldots, \sigma_n): f(\sigma_1, \ldots, \sigma_n) = 1}x_1^{\sigma_1}\ldots x_n^{\sigma_n}.\]
\end{corollary}

\begin{theorem}
    Каждая функция алгебры логики может быть получена суперпозициями из отрицания, конъюнкции и дизъюнкции.
\end{theorem}

\begin{proof}
    Если функция не тождественно нулевая, то она реализуется с помощью СДНФ. Её можно рассматривать как формулу алгебры логики, построенную при помощи отрицаний, конъюнкций и дизъюнкций. Если функция тождественно нулевая, то её можно задать формулой $x_1 \cdot \overline{x}_1$, рассматриваемой относительно списка фиктивных переменных требуемой длины.
\end{proof}

\begin{theorem}
    $\forall f \in P_2(n)$, $f\neq 1$ справедлива формула (\textit{совершенная конъюнктивная нормальная форма}).
    \[
    f(x_1, \ldots, x_n) = \prod_{(\delta_1, \ldots, \delta_n): f(\delta_1, \ldots, \delta_n) = 0}(x_1^{\overline{\delta_1}} \vee \ldots\vee x_n^{\overline{\delta_n}}).
\]
\end{theorem}

\begin{proof}
    Применяя разложение в виде СДНФ для $\overline{f(x_1, \ldots, x_n)}$, имеем
    \begin{equation*}
        \begin{split}
        f(x_1, \ldots, x_n) &= \overline{\overline{f(x_1, \ldots, x_n)}} = \overline{\bigvee\limits_{(\delta_1, \ldots, \delta_n): \overline{f(\delta_1, \ldots, \delta_n)} = 1}x_1^{\delta_1}\ldots x_n^{\delta_n}} = \prod_{(\delta_1, \ldots, \delta_n): f(\delta_1, \ldots, \delta_n) = 0}\overline{(x_1^{\delta_1}\ldots x_n^{\delta_n})} = \\
        &= \prod_{(\delta_1, \ldots, \delta_n): f(\delta_1, \ldots, \delta_n) = 0}(\overline{x_1^{\delta_1}} \vee \ldots\vee \overline{x_n^{\delta_n}}) = \prod_{(\delta_1, \ldots, \delta_n): f(\delta_1, \ldots, \delta_n) = 0}(x_1^{\overline{\delta_1}} \vee \ldots\vee x_n^{\overline{\delta_n}}).
        \end{split}
    \end{equation*}
\end{proof}