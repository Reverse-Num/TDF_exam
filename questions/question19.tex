\section{Пример Янова замкнутого класса $k$-значной логики, не имеющего базиса.}

\begin{definition}
    Множество $B$ функций $k$-значной логики называется \textit{базисом} в замкнутом классе $F$, если выполняются два условия:
    \begin{enumerate}[nolistsep]
        \item $[B] = F$;
        \item для любого собственного подмножества $A$ множества $B$ равенство $[A] = F$ не выполняется.
    \end{enumerate}
\end{definition}

\begin{theorem}[Ю.\,И\,Янов]
    При $k \geqslant 3$ в $P_k$ существует замкнутый класс, не имеющий базиса.
\end{theorem}

\begin{proof}
    Рассмотрим следующую систему функций $F = \{f_0, \ldots, f_n, \ldots\}$, где $f_0 = 0$ и при любом $n\geqslant 1$
    \[
        f_n(x_1, \ldots, x_n) = 
        \begin{cases}
            1,&\text{если $x_1 = \ldots = x_n = 2$};\\
            0,&\text{иначе.}
        \end{cases}
    \]
    Положим $M_k = [F]$. Любая нетривиальная суперпозиция равна 0, поэтому каждая функция из $M_k$ получается из функций системы $F$ подстановкой переменных. При $n > m$, $f_m$ получается из $f_n$ отождествлением некоторых переменных, поэтому в базисе класса $M_k$ не может быть более одной функции, но и одной быть не может, т.к $f_n$ нельзя выразить через $f_m$.
\end{proof}
