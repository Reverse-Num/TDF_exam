\section{Метод каскадов получения верхней оценки функции Шеннона.}

\begin{theorem}[\texttt{Метод каскадов}]
    При $n \to \infty$ верна следующая верхняя оценка функции Шеннона
    \[
        L(n) \lesssim 6 \frac{2^n}{n}.
    \]
\end{theorem}

\begin{proof}
    Последовательно определим множества функций $G_0, G_1, \ldots, G_{n-1}$, имеющие вид $G_i = \{g_{i,1}(x_{i+1}, \ldots, x_n), \ldots, g_{i,r_i}(x_{i+1}, \ldots, x_n)\}$ следующие образом:
    \begin{enumerate}[nolistsep]
        \item $G_0 = \{g_{0,1} (x_1, \ldots, x_n)\} = \{f(x_1, \ldots, x_n)\}$.
        \item Если определено множество $G_{i-1} = \{g_{i-1,1}(x_{i}, \ldots, x_n), \ldots, g_{i-1,r_{i-1}}(x_{i}, \ldots, x_n)\}$, то 
        \begin{multline*}
            G_i = \{g_{i-1,1}(0, x_{i+1}, \ldots, x_n), \ldots, g_{i-1,r_{i-1}}(0, x_{i+1}, \ldots, x_n),\\
            g_{i-1,1}(1, x_{i+1}, \ldots, x_n), \ldots, g_{i-1,r_{i-1}}(1, x_{i+1}, \ldots, x_n)\}.
        \end{multline*}
    \end{enumerate}
    Отметим некоторые свойства множеств $G_i$:
    \begin{enumerate}
        \item Для любой функции из $g$ из $G_{i-1}$ найдутся такие функции $g^{(1)}$ и $g^{(2)}$ из $G_{i-1}$, что справедливо равенство $g(x_i, \ldots, x_n) = x_ig^{(1)}(x_{i+1}, \ldots, x_n) \vee \overline{x}_i g^{(2)}(x_{i+1}, \ldots, x_n)$.
        \item При $i=1,\ldots, n-1$ для количества $r_i$ элементов множества $G_i$ выполняются неравенства: $r_i \leqslant 2 r_{i-1} \leqslant 2^i$ (в силу построения $G_i$), и $r_{n-i} \leqslant 2^{2^i}$ поскольку $G_{n-i}$ содержит функции только от $i$ переменных. 
    \end{enumerate}
    Перейдём к описанию схемы, последовательно реализующей эти множества.

    Так как $G_{n-1} \subseteq {0, 1, x_n, \overline{x}_n}$, то для реализации функций из множества $G_{n-1}$ потребуется не более трех элементов.

    После этого вычислим отрицания всех остальных переменных, затратив еще $n - 1$ инвертор (т.\,е. элемент, реализующий отрицание функции, подаваемой на единственный вход этого элемента).

    Далее, если уже реализованы все функции из множества $G_i$, то для вычисления любой функции из множества $G_{i-1}$ в соотвествии со свойством 1 достаточно трёх элементов (2 конъюнкции и дизъюнкция). Поэтому
    \[
        L(f) \leqslant 3 + (n-1) + \sum_{i=0}^{n-2}3r_i.
    \]
    Введём натуральный параметр $k$. Для оценки сверху величины $r_i$ в зависимости от выполнения условия $i \leqslant n - k - 1$ применим разные оценки из свойства 2:
    \[
        L(f) \leqslant n+2 + 3(1+2+\ldots + 2^{n-k-1}) + 3(2^{2^k}+\ldots + 2^{2^1}) \leqslant n + 3\cdot 2^{n-k} + 6\cdot 2^{2^k}.
    \]
    Полагая $k = \floor{\log (n - 2\log n)}$, получаем $\dfrac{n-2\log n}{2} < 2^k \leqslant n - 2\log n$, а значит
    \[
        L(n) \leqslant 6 \frac{2^n}{n} + o\left(\frac{2^n}{n}\right).
     \]
\end{proof}