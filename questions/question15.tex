\section{Полнота системы $\{\max(x, y), \overline{x}\}$ в $P_k$. Функция Вебба. Полнота системы, состоящей только из функции Вебба}

\begin{theorem}
    Система $\{\max(x, y), \overline{x}\}$ полна в $P_k$.
\end{theorem}

\begin{proof}
    \begin{enumerate}
        \item Получение констант.
        Построим сначала константы, начиная с $k - 1$:
        \[
            k-1 = \max(x, \ldots, x+k-1).
        \]
        Остальные константы получаются при помощи функции $x + 1 \pmod k$.
        \item Построение функций $I_i(x), i = 0, 1, \ldots, k - 1$:
        $I_0(x) = \max(x, x+1, \ldots, x+k-2) + 1$;
        \[
            I_i(x) = I_0(x - i).
        \]
        \item Получение функции $\min(x, y)$ при наличии функции $\sim x$:
        \[
            \min(x, y) = \;\sim (\max(\sim x, \sim y)).
        \]
        \item Построение произвольной функции $g(x)$ одной переменной.
        \begin{enumerate}
            \item Для произвольных $\alpha$, $\beta$ из $E_k$ построим функцию 
            $
            \varphi_{\alpha, \beta}(x) = 
            \begin{cases}
            \beta,&\text{если $x = \alpha$},\\
                0,&\text{иначе}
            \end{cases}
            $, используя формулу:
            \[
                \varphi_{\alpha, \beta}(x) = \max(I_\alpha(x), k-1-\beta) + \beta + 1.
            \]
            \item Представление функции $g(x)$:
            \[
                g(x) = \max(\varphi_{0, g(0)}(x), \varphi_{1, g(1)}(x), \ldots \varphi_{k-1, g(k-1)}(x)).
            \]
        \end{enumerate}
    \end{enumerate}
    Таким образом, получем $\sim x$ и выражение системы из теоремы 3 через систему $\{\max(x, y), \overline{x}\}$.
\end{proof}

\begin{definition}
    $V_k(x, y) \vcentcolon = \max(x, y) + 1$ --- \textit{функция Вебба}.
\end{definition}

\begin{theorem}
    Система $\{V_k(x, y)\}$ полна в $P_k$.
\end{theorem}

\begin{proof}
    Из функции Вебба получается отрицание Поста $V_k(x, x) = x + 1 = \overline{x}$. Следовательно, получаем функции $x + i$ ($i = 0, \ldots, k - 1$). Теперь получаем $\max(x, y) = V_k(x, y) + (k - 1)$. Имеем всю полную систему $\{\overline{x}, \max(x, y)\}$.
\end{proof}