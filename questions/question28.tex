\section{Метод Шеннона получения верхней оценки функции Шеннона.}

\begin{theorem}[\texttt{Метод Шенона}]
    При $n \to \infty$ верна следующая верхняя оценка функции Шеннона
    \[
        L(n) \lesssim 6 \frac{2^n}{n}.
    \]
\end{theorem}

\begin{proof}
    Пусть $k$ --- натуральный параметр, значение которого выберем позже. Разложим функцию $f(x_1, \ldots, x_n)$ по первым $n-k$ переменным:
    \[
        f(x_1, \ldots, x_n) = \bigvee\limits_{(\sigma_1, \ldots, \sigma_{n-k})} x_1^{\sigma_1}\cdot \ldots \cdot x_{n-k}^{\sigma_{n-k}} f(\sigma_1, \ldots, \sigma_{n-k}, x_{n-k+1}, \ldots, x_n).
    \]
    Построим схему, реализующую функцию $f$ как показано на рисунке: cхема состоит из трех блоков (подсхем).
    %Всё ещё лень переделывать. Смотри рисунок на в билете 27
    \begin{center}
        \begin{tikzpicture}[
            smallrectangle/.style={rectangle, draw=red!60, fill=red!5, very thick, minimum size=5mm, minimum width=4cm, minimum height=0.8cm}, bigrectangle/.style={rectangle, draw=blue!60, fill=blue!5, very thick, minimum size=5mm, minimum width=9cm, minimum height=0.8cm}
            ]
            %Nodes
            \node[bigrectangle](big_rectangle) at (4.5,-1.6) {<<$\&$>> + <<$\vee$>>};
            \node[smallrectangle](left_rectangle) at (2,0)  {$K_{n-k}(x_1, \ldots, x_{n-k})$};
            \node[smallrectangle](right_rectangle) at (7,0) {$P_2(x_{n-k+1}, \ldots, x_n  )$};
            \node (x1) at (0.5, 1.2) {$x_1$};
            \node (...1) at (2, 1.2) {$\ldots \quad \ldots$};
            \node (xn/2-1) at (3.5, 1.2) {$x_{n-k}$};
            \node (xn/2+1) at (5.5, 1.2) {$x_{n-k+1}$};
            \node (...2) at (7, 1.2) {$\ldots \quad \ldots$};
            \node (xn) at (8.5, 1.2) {$x_n$};
            \node (f) at (4.5, -2.8) {$f$};
            
            %Lines
            \draw[arrows = {-Latex[width=3pt, length=3pt]}] (left_rectangle.south) -- (2, -1.2);
            \draw[arrows = {-Latex[width=3pt, length=3pt]}] (right_rectangle.south) -- (7, -1.2);
            \draw[arrows = {Circle[length=3pt, width=3pt, open]}-] (x1) -- (0.5, 0.4);
            \draw[arrows = {Circle[length=3pt, width=3pt, open]}-] (xn/2-1) -- (3.5, 0.4);
            \draw[arrows = {Circle[length=3pt, width=3pt, open]}-] (xn/2+1) -- (5.5, 0.4);
            \draw[arrows = {Circle[length=3pt, width=3pt, open]}-] (xn) -- (8.5, 0.4);
            \draw[arrows = {-Circle[length=3pt, width=3pt]}] (4.5, -2) -- (f);
        \end{tikzpicture}
    \end{center}
     Первая подсхема по переменным $x_1, \ldots, x_{n-k}$ реализует систему конъюнкций $K_{n-k}(x_1, \ldots, x_{n-k})$, вторая по переменным $x_{n-k+1}, \ldots, x_n$ реализует систему всех $2^{2^k}$ функций от этих $k$ переменных, а третья в соответствии с указанным разложением функции $f$ реализует саму функцию $f$. Третий блок реализуется не более, чем за $2\cdot 2^{n-k}$ операций ($2^{n-k}$ конъюнкций, и $2^{n-k}-1$ дизъюнкций).

     Применяя теоремы из билетов 26 и 27, получаем:
     \[
        L(n) = L(f) \leqslant L(K_{n-k}) + L(k)2^{2^k} + 2\cdot 2^{n-k} \leqslant 2^{n-k} + o(2^{n-k}) + k2^k2^{2^k} + 2\cdot 2^{n-k} = \frac{3\cdot 2^n}{2^k} + o\left(\frac{2^n}{2^k}\right) + k2^k2^{2^k}.
     \]
     Полагая $k = \floor{\log(n-3\log n)}$, имеем $\dfrac{n-3\log n}{2} < 2^k \leqslant n-3\log n$.
     Окончательно получаем:
     \[
        L(n) \leqslant 6 \frac{2^n}{n} + o\left(\frac{2^n}{n}\right).
     \]
\end{proof}