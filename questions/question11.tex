\section{Критерий Поста полноты множества функций в $P_2$. Следствие о существовании в любом полном множестве полного подмножества из не более чем 4 функций. Пример базиса в $P_2$, состоящего из четырех функций.}

\begin{theorem}[Критерий Поста]
    Система функций из $P_2$ полна тогда и только тогда, когда она не содержится ни в одном из классов $T_0$, $T_1$, $L$, $S$, $M$.
\end{theorem}

\begin{proof}
    $\Rightarrow$. Пусть $F$ полна и $F \subseteq K$, где $K \in \{T_0, T_1, L, S, M\}$. Тогда $[F] \subseteq [K] = K \ne P_2$.

    $\Leftarrow$. Обратно, пусть $F \not\subseteq T_0$, $F \not\subseteq T_1$, $F \not\subseteq L$, $F \not\subseteq S$, $F \not\subseteq M$. Тогда в $F$ найдутся $f_0 \notin T_0$, $f_1 \notin T_1$, $f_L \notin L$, $f_S \notin S$, $f_M \notin M$. Возможны два случая:
    \begin{enumerate}
        \item $f_0 \in T_1$. Тогда $\varphi(x) \vcentcolon = f_0(x, \ldots, x) = 1$. Так получаем константу $1$. Чтобы получить константу $0$, достаточно теперь воспользоваться функцией $f_1$. Теперь по лемме о немонотонной функции, из $f_M$, $0$, $1$ можно получить $\overline{x}$. 
        \item $f_0 \notin T_1$. Тогда $\varphi(x) \vcentcolon = f_1(x, \ldots, x) = \overline{x}$. По лемме о несамодвойственной функции, из $f_S$ и $\overline{x}$ можно получить константу. Имея отрицание, получаем также и другую константу.
    \end{enumerate}

    По лемме о нелинейной функции, из $f_L$, $0$, $1$ и $\overline{x}$ можно получить $x_1 \cdot x_2$. Следовательно, из $F$ можно выразить полную систему $\{\overline{x}, x_1 \cdot x_2\}$, поэтому $F$ также полна.
\end{proof}

\begin{corollary}
    Каждый замкнутый класс функций из $P_2$, отличный от $P_2$, содержится хотя бы в одном из классов $T_0$, $T_1$, $L$, $S$, $M$.
\end{corollary}

\begin{proof}
    Действительно, если бы он не содержался ни в одном из этих классов, то был бы полон, а т.\,к. замкнут, то совпал бы с $P_2$.
\end{proof}

\begin{corollary}
    В любой полной системе существует полная подсистема, состоящая не более, чем из 4 функций.
\end{corollary}

\begin{proof}
    В первом случае доказательства нужно брать $f_0$, $f_1$, $f_M$, $f_L$, а во втором $f_0$, $f_1$, $f_S$, $f_L$.
\end{proof}

\begin{example}[Полная система из 4 функций]
    $\{0, 1, x \oplus y \oplus z, x \& y\}$.
\end{example}