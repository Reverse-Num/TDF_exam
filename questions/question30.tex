\section{Точное значение сложности реализации системы всех функций от $n$ переменных в произвольном полном базисе.}

\begin{theorem}
    Пусть $[B] = P_2$. Тогда для любого натурального $n$ справедливо равенство
    \[
        L_B(P_2(x_1, \ldots, x_n)) = 2^{2^n} - n.
    \]
\end{theorem}

\begin{proof}
    Оценка снизу очевидна (для каждой нетривиальной функции, необходим хотя бы один ФЭ), оценим сверху. Рассмотрим произвольную схему, реализующую $P_2(n)$. Если два функциональных элемента будут реализовывать одинаковые функции, удалим один ФЭ и все исходящие из него рёбра заменим на те же рёбра, выходящие из другого ФЭ. Тем самым, можно добиться того, что каждой нетривиальной функции из $P_2(n)$ будет сопоставлен один ФЭ, а значит
    \[
        L_B(P_2(x_1, \ldots, x_n)) \leqslant 2^{2^n} - n.
    \]
\end{proof}

\begin{lemma}
    Пусть $B_1$ и $B_2$ --- конечные множества БФ, такие, что $[B_1] = [B_2] = P_2$, тогда существуют $c_1$, $c_2$ $>0$ такие, что $\forall f \in P_2$ справедливо неравенство $c_1L_{B_1}(f)\leqslant L_{B_2}(f) \leqslant c_2L_{B_1}(f)$.
\end{lemma}

\begin{proof}
    Очевидно верно для $\ds c_2 = \max_{\varphi \in B_1} L_{B_2}(\varphi), c_1 = \left(\max_{\varphi \in B_2} L_{B_1}(\varphi)\right)^{-1}$.
\end{proof}