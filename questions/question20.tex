\section{Пример Мучника замкнутого класса $k$-значной логики со счетным базисом. Континуальность семейства замкнутых классов функций $k$-значной логики.}

\begin{theorem}[A.\,A.\,Мучник]
    При $k \geqslant 3$ семейство замкнутых классов функций $k$-значной логики континуально.
\end{theorem}

\begin{proof}
     Количество замкнутых классов не более числа различных подмножеств счетного множества, т.\,е. семейство замкнутых классов функций $k$-значной логики не более чем континуально.

     Построим континуальное семейство замкнутых классов. Для $s = 2, 3, \ldots$ определим (симметрическую) функцию $f_s(x_1, \ldots, x_s)$ следующим образом. Функция $f_s$ принимает значение 1 на наборах, состоящих из $s - 1$ двойки и одной единицы, и значение 0 на всех остальных наборах.

     Положим $F = \displaystyle \bigcup_{i=2}^\infty\{f_i\}$. Покажем, что $f_s\notin [F \setminus \{f_s\}]$ для любого $s\geqslant 2$. Пусть это не так, т.\,е. 
     \[
        f_s(x_1,\ldots,x_s) = f_n(A_1, \ldots, A_n).
     \]
     Возможны три случая:
     \begin{enumerate}
         \item Если среди $A_i$ хотя бы две нетривиальные формулы, то $f_n(A_1, \ldots, A_n)$ --- тождественный 0, противоречие.
         \item Если среди $A_i$ одна нетривиальная формула (пусть это $i$-ая формула), то найдётся тривиальная формула $A_j = x_j$, тогда если $x_j = 1$, а оставшиеся переменные примем за 2, получим
         \[
            f_s(2, \ldots, 1, \ldots,2) = 1 \neq 0 = f_n(2, \ldots, 1, \ldots, A_i, \ldots, 2),
         \]
         где $A_i$ нетривиальная, поэтому на этом наборе равна $1$ или $0$. Противоречие.
         \item Если среди $A_i$ все формулы тривиальные, то т.\,к. $n > s$, хотя бы одна переменная $x_j$ встретится дважды, тогда если $x_j = 1$, а оставшиеся переменные примем за 2, получим
         \[
            f_s(2, \ldots, 1, \ldots,2) = 1 \neq 0 = f_n(2, \ldots, 1, \ldots, 1, \ldots, 2),
         \]
         что приводит к противоречию.
     \end{enumerate}
     Осталось построить континуальное семейство замкнутых классов. Рассмотрим $R$ --- множество всех последовательностей 0 и 1. Для произвольной последовательности $\widetilde{\alpha} = (\alpha_1, \ldots, \alpha_s, \ldots)$ положим 
     \[
        F_{\widetilde{\alpha}} = \bigcup_{i:\, \alpha_i = 1} \{f_{i+1}\},
     \]
     т.\,е. если на $i$-ом месте последовательности стоит 1, мы добавляем в систему $f_{i+1}$.

     Если $\widetilde{\alpha}$, $\widetilde{\beta}$ --- различные последовательности, то $[F_{\widetilde{\alpha}}] \neq [F_{\widetilde{\beta}}]$, потому что найдётся функция $f_\gamma$, которая есть в одной системе и которой нет в другой, и, как было показано ранее, её нельзя выразить через другую систему. Т.\,к. множество двоичных последовательностей континуально, получаем, что и семейство замкнутых классов $\{[F_{\widetilde{\alpha}}]: \widetilde{\alpha} \in R\}$ континуально.
\end{proof}