\section{Алгоритм распознавания полноты системы функций $k$-значной логики. Исследование полноты систем функций $k$-значной логики на практике.}

\begin{definition}
    Пусть $F$ --- произвольное множество функций из $P_k$ , а $\widetilde{x}$ --- набор переменных $(x_1, \ldots, x_p)$, $p \geqslant 1$. Через $F[\widetilde{x}]$ или через $F[x_1, \ldots, x_p]$ обозначим множество функций из $F$, зависящих от переменных $x_1, \ldots, x_p$.
\end{definition}

\begin{theorem}
    Существует алгоритм распознавания полноты конечных систем функций из $P_k$.
\end{theorem}

\begin{proof}
    Пусть $F = \{f^1(x_1, \ldots, x_{n_1}), \ldots, f^t(x_1,\ldots, x_{n_t})\} \subset P_k$. Последовательно построим множества функций $R_0$, $R_1$, $\ldots$ из $P_k[x_1, x_2]$:
    \begin{enumerate}
        \item $R_0 = \varnothing$;
        \item Если построено множество $R_s$, то множество $R_{s+1}$ --- это множество всех функций, задаваемых формулами вида $f^i(A_1, \ldots, A_{n_i})$, где $f^i \in F$, а $A_j$ либо $x_1, x_2$, либо функция из $R_s$. 
    \end{enumerate}
    Последовательность $R_0 \subseteq R_1 \subseteq \ldots \subseteq R_s \subseteq R_{s+1} \subseteq \ldots $ в какой-то момент стабилизируется (в силу ограниченности количества функций от 2-x переменных), т.\,е. найдётся такое $r$, что 
    \[
        R_r = R_{r+1} = \ldots.
    \]
    Система $F$ будет полной в том и только том случае, когда $V_k(x_1, x_2) \in R_r$. В самом деле, если $V_k(x_1, x_2) \in R_r$, то значит суперпозициями над $F$ можно получить полную систему, а значит система $F$ полна. И обратно, если $V_k(x_1, x_2) \notin R_r$, значит суперпозициями над $F$ нельзя получить функцию из $P_k$, а значит, $F$ неполна.
\end{proof}