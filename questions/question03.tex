\section{Формулы над множеством булевых функций. Реализация булевых функций формулами. Операция суперпозиции. Эквивалентность формул.}
Пусть дано множество переменных $X$, а также (конечное или счётное) множество функциональных символов $A=\{f_1^{(n_1)}, \ldots, f_k^{(n_k)},\ldots\}$.
\begin{definition}
    Определим \textit{формулу над множеством переменных $X$ и множеством функциональных символов $A$} индуктивно:
    \begin{enumerate}[nolistsep]
        \item Любая переменная из $X$ --- формлуа;
        \item Если $\Phi_1,\ldots ,\Phi_{n_i}$ --- формулы, то $f_i^{(n_i)}(\Phi_1, \ldots, \Phi_{n_i})$ --- также формула.
    \end{enumerate}
\end{definition}
\textit{Формула над множеством переменных $X$ и множеством функциональных символов $A$} --- это
последовательность из функциональных символов, символов
переменных, скобок и запятых, которую можно получить по
указанным правилам за конечное число шагов.

Формула --- не функция, а последовательность символов.
Однако каждая формула задает функцию: естественным образом определяется значение формулы на каждом наборе значенимй переменных, входящих в формулу. Или же более формально:
\begin{definition}
    Определим \textit{значение формулы $\Phi$ на наборе $\widetilde{\alpha}$} переменных $\widetilde{x}$ индукцией по построению формулы $\Phi$:
    \begin{enumerate}[nolistsep]
        \item Если $\Phi$ есть однобуквенное слово $x_{i}$, то $\Phi[\widetilde{x}, \widetilde{\alpha}] \vcentcolon = \alpha_i$.
        \item Пусть $\Phi$ имеет вид $f(\Phi_1, \ldots, \Phi_m)$, причём $\Phi_1[\widetilde{x}, \widetilde{\alpha}] = \beta_1, \ldots, \Phi_m[\widetilde{x}, \widetilde{\alpha}] = \beta_m$. \\Тогда $\Phi[\widetilde{x}, \widetilde{\alpha}] \vcentcolon = f(\beta_1, \ldots, \beta_m)$.
    \end{enumerate}
\end{definition}

\begin{definition}
    Формулы, реализующие равные функции, называются \textit{эквивалентными}.
\end{definition}

\begin{definition}
    Если функция $f(x_1, \ldots, x_n)$ реализуется нетривиальной формулой над системой функций $F$, то говорят, что она получена операцией суперпозиции из функций системы $F$.
\end{definition}