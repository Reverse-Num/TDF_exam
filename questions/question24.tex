\section{Ориентированные графы без ориентированных циклов. Лемма о правильной (монотонной) нумерации вершин в конечном ориентированном графе без циклов.}

\begin{definition}
    \textit{Ориентированный цикл} --- цикл, в котором все рёбра ориентированные.
\end{definition}

\begin{definition}
    Нумерацию вершин в конечном ориентированном графе без ориентированных циклов первыми идущими подряд натуральными числами будем называть \textit{монотонной} или \textit{правильной}, если, любое ребро направлено от вершины с меньшим номером к вершине с большим. 
\end{definition}

\begin{lemma}[О монотонной нумерации вершин]
    У любого конечного ориентированного графа без ориентированных циклов существует монотонная нумерация.
\end{lemma}

\begin{proof}
    Докажем индукцией по $n$ --- колиечству вершин. База $n=1$ --- верно. Пусть в графе $n$ вершин. Найдём такую, из которой не выходит ни одного ребра (так можно сделать, потому что граф без ориентированных циклов), сопоставим ей номер $n$. Остальные $n-1$ вершин можно занумеровать по предположению. Получена монотонная нумерация для $n$ вершин.
\end{proof}