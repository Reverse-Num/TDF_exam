\section{Канонические уравнения автомата в скалярном (булевом) виде. Операция обратной связи. Конечные полные системы автоматных функций относительно операций cуперпозиции и обратной связи. Реализация автомата схемами из функциональных элементов и элементов задержки.}

Отметим, что функциональные элементы, можно рассматривать как автоматы с одним состоянием.

\begin{definition}
    Канонические уравнения для автоматов, соответствующих элементам дизъюнкции, конъюнкции и отрицания, имеют следующий вид:
    \[
        \begin{cases}
            y(t) = x_1(t)\vee x_2(t),\\
            q(t+1) = q(t),\\
            q(0)=0,
        \end{cases}
        \quad
        \begin{cases}
            y(t) = x_1(t)\&x_2(t),\\
            q(t+1) = q(t),\\
            q(0)=0,
        \end{cases}
        \quad
        \begin{cases}
            y(t) = \overline{x(t)},\\
            q(t+1) = q(t),\\
            q(0)=0,
        \end{cases}
    \]
    Будем обозначать автоматные функции из $P_E$, где $E=\{0, 1\}$, которые вычисляются этими автоматами через $f_\vee(x_1, x_2), f_\&(x_1,x_2), f_\_(x)$.
\end{definition}

\begin{definition}
    \textit{Элемент единичной задержки} --- автомат с каноническим уравнением:
    \[
    \begin{cases}
            y(t) = q(t),\\
            q(t+1) = x(t),\\
            q(0)=0,
    \end{cases}
    \]
    соответствующая автоматная функция обозначается через $\vec{f}(x)$.
\end{definition}

Рассмотрим произвольную автоматную функцию $f(x_1, \ldots, x_n)$ из $P_E$ и инициальный автомат $V_{q_0} = (A, B, Q, F, G_{q_0})$, вычисляющий её, где $A=E^n$, $B=E$, $Q=\{q_1, \ldots, q_\lambda\}$, $q_0 \in Q$, $F:E^n \times Q \to E$, $G:E^n \times Q \to Q$ с каноническими уравнениями:
\[
    \begin{cases}
        y(t) = F(x_1(t), \ldots, x_n(t), q(t)),\\
        q(t+1) = G(x_1(t), \ldots, x_n(t), q(t)),\\
        q(0) = q_0.
    \end{cases}
\]

\begin{definition}
    Положим $l = \ceil{\log \lambda}$. Занумеруем состояния $q_1, \ldots, q_\lambda$ наборами 0 и 1, причём начальному состоянию $q_0$ сопоставим набор $(0, \ldots, 0)$, т.\,е. состояние автомата $q(t)$ в момент времени $t$ будет кодироваться $l$ двоичными состояниями: $(q_1(t), \ldots, q_l(t))$. Рассмотрим новые функции $F^1:E^{n+l}\to B$, $G^1 = (G_1,\ldots, G_l)$, где $G_i: E^{n+l}\to E$, $G^1:E^{n+l}\to E^l$, т.\,е.
    \[
        \begin{cases}
            y(t) = F^1(x_1(t), \ldots, x_n(t), q_1(t), \ldots, q_l(t)),\\
            (q_1(t+1), \ldots, q_l(t+1)) = G^1(x_1(t), \ldots, x_n(t), q_1(t), \ldots, q_l(t)),\\
            (q_1(0), \ldots, q_l(0)) = (0,\ldots, 0).
        \end{cases}
    \]
    Или же в общем виде:
    \[
        \begin{cases}
            y(t) = F^1(x_1(t), \ldots, x_n(t), q_1(t), \ldots, q_l(t)),\\
            q_1(t+1) = G_1(x_1(t), \ldots, x_n(t), q_1(t), \ldots, q_l(t)),\\
            \ldots\\
            q_l(t+1) = G_l(x_1(t), \ldots, x_n(t), q_1(t), \ldots, q_l(t)),\\
            q_1(0) = 0,\\
            \ldots\\
            q_l(0) = 0.
        \end{cases}
    \]
    Эти уравнения называются \textit{каноническими уравнениями автомата в скалярном} (\textit{булевом}) \textit{виде}. Таким образом мы построили инициальный автомат $V^1_{\widetilde{q_0}} = (E^n, E, E^l, F^1, G^1)$.
\end{definition}

Рассмотрим СФЭ в базисе $\{\vee, \&, \overline{x}\}$ $S$ c входами $x_1, \ldots, x_n, q_1, \ldots, q_l$ и с выходами $y, z_1, \ldots, z_l$, где $z_i = G_i(x_1, \ldots, x_n, q_1, \ldots, q_l)$. Преобразуем $S$ следующим образом. Соединим $z_i$ c $q_i$, добавив элемент единичной задержки, тем самым добавив $l$ элементов единичной задержки. Кроме того, заменим $x_i$ на $x_i(t)$, $q_j$ на $q_j(t)$, $y$ на $y(t)$, а ФЭ $\vee, \&, \overline{x}$ на символы $f_\vee, f_\&, f_\_$ соответственно. В результате получим схему $S_1$ из автоматных элементов в базисе $f_\vee, f_\&, f_\_, \vec{f}$.

\begin{remark}
    В схеме $S_1$ мы вышли за рамки данного ранее определения схемы из автоматных элементов, поскольку появились ориентированные циклы, однако очевидно (по индукции), что если в момент времени $t$ подавать на входы схемы $S_1$ значения $x_1(t), \ldots, x_n(t)$, то на её выходе будет выдаваться значение $y(t)$, которое вычисляется в соответствии с каноническими уравнениями $V^1_{\widetilde{q_0}}$. А значит, схема $S_1$ реализует автоматную функцию $f(x_1, \ldots, x_n)$.
\end{remark}

\begin{definition}
    Операция построения в схемах ориентированных циклов, проходящих через элементы задержки, называется \textit{операцией обратной связи}.
\end{definition}

\begin{theorem}
    Любую автоматную функцию из $P_E$ можно реализовать схемой из автоматных элементов в базисе $\{f_\vee, f_\&, f_\_, \vec{f}\}$ с использованием операции обратной связи.
\end{theorem}

\begin{proof}
    Прямое следствие построения нового инициального автомата по каноническим уравнениям автомата в скалярном виде и предыдущих замечаний.
\end{proof}