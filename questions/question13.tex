\section{Функции $k$-значной логики ($k \geqslant 3$). Число функций $k$-значной логики от $n$ фиксированных переменных. Существенные и фиктивные переменные для функций $k$-значной логики, отличие от случая булевых функций. Элементарные функции $k$-значной логики, их свойства.}

\begin{definition}
    $E_k \vcentcolon = \{0, 1, \ldots, k - 1\}$.
\end{definition}

\begin{definition}
    Функцию $f(x_1, \ldots, x_n): \displaystyle E_k^n \to E_k$ будем называть \textit{функцией $k$-значной логики}. Множество всех таких функций обозначается как $P_k$.
\end{definition}

\begin{theorem}
    Число функций $k$-значной логики от $n$ переменных равно $k^{k^n}$.
\end{theorem}

\begin{proof}
    В самом деле, наша функция определяется значениями, которая она принимает на $k^n$ наборах. Для каждого набора $k$ значений, а значит, способов выбрать значения $k^{k^n}$.
\end{proof}

\begin{definition}
    Функция $k$-значной логики $f(\widetilde{x}^n)$ называется \textit{существенно зависящей от переменной $x_i$} ($i = 1, \ldots, n$), если существуют значения $\sigma_1, \ldots, \sigma_{i - 1}, \sigma_{i + 1}, \ldots, \sigma_n$, $\sigma^\prime$, $\sigma^{\prime\prime}$ из $E_k$ такие, что \[f(\sigma_1, \ldots, \sigma_{i - 1}, \sigma^\prime, \sigma_{i + 1}, \ldots, \sigma_n) \ne f(\sigma_1, \ldots, \sigma_{i - 1}, \sigma^{\prime\prime}, \sigma_{i + 1}, \ldots, \sigma_n).\]
    В этом случае $x_i$ называется \textit{существенной переменной функции $f$}. Переменная, не являющаяся существенной называется \textit{фиктивной}.
\end{definition}

\begin{definition}
    Пусть $x_i$ --- фиктивная переменная функции $f(\widetilde{x}^n)$. Тогда функция \[g(x_1, \ldots, x_{i - 1}, x_{i + 1}, \ldots, x_n) \vcentcolon = f(x_1, \ldots, x_{i - 1}, 0, x_{i + 1}, \ldots, x_n)\]
    называется \textit{полученной из $f$ удалением фиктивной переменной $x_i$}. Обратно, говорят, что \textit{$f$ получена из $g$ добавлением $i$-ой фиктивной переменной}.
\end{definition}

\begin{definition}
    Две функции $k$-значной логики $f$ и $g$ называются \textit{одинаковыми}, если у них одинаковое множество переменных и на любом наборе этих переменных функции принимают одинаковые значения.
\end{definition}

\begin{definition}
    Две функции $k$-значной логики $f$ и $g$ называются \textit{равными}, если одну из другой можно получить за конечное число применений операций добавления и удаления фиктивных переменных.
\end{definition}

Отличие от случая булевых функций заключается в том, что при подстановке одной функции в другую существенная зависимость переменных не сохраняется, в отличие от $P_2$. Достаточно привести пример:
\begin{example}\vspace{-8mm}\\
    \begin{minipage}[t]{0.76\textwidth}
        \parindent=12pt % Устанавливаем отступ 12pt
        Функция $\varphi(x,y)$ существенно зависит от переменных $x$, $y$.
        
        Однако $\varphi(x, \varphi(x,y))$ --- тождественный ноль.
    \end{minipage}
    \begin{minipage}[t]{0.2\textwidth}
        \centering
        \begin{tabular}{c|c|c|c}
            \backslashbox{$x$}{$y$} & 0 & 1 & 2 \\
            \hline
            0 & 0 & 0 & 0\\
            \hline
            1 & 0 & 0 & 0\\
            \hline
            2 & 0 & 0 & 1\\
        \end{tabular}
    \end{minipage}
\end{example}

Как и в случае алгебры логики, выделяется некоторый \textbf{список элементарных функций}:
\begin{enumerate}[nolistsep]
    \item \textit{Константы} $0, 1, \ldots, k - 1$ и \textit{тождественная функция} $x$.
    \item $\overline{x} \vcentcolon = x + 1 \pmod k$ --- \textit{отрицание Поста}.
    \item ${\sim}x \vcentcolon = k - 1 - x$ --- \textit{отрицание Лукашевича}
    \item 
        $
        I_\sigma(x) \vcentcolon =
        \begin{cases}
            k - 1,&\text{если $x = \sigma$},\\
            0,&\text{иначе}
        \end{cases}
        $ --- \textit{индикаторная функция}, принимающая в $\sigma$ <<большое значение>>.
    \item 
        $
        j_\sigma(x) \vcentcolon =
        \begin{cases}
            1,&\text{если $x = \sigma$},\\
            0,&\text{иначе}
        \end{cases}
        $ --- \textit{индикаторная функция}, принимающая в $\sigma$ <<маленькое значение>>.
    \item $\min(x, y)$ --- возможное обобщение конъюнкции (часто будем обозначать через $x \& y$).
    \item $x \cdot y \pmod k$ --- другое возможное обобщение конъюнкции.
    \item $\max(x, y)$ --- возможное обобщение дизъюнкции (часто будем обозначать через $x \vee y$).
    \item $x + y \pmod k$ --- другое возможное обобщение дизъюнкции.
\end{enumerate}

Отметим следующие \textbf{свойства операций}:

\begin{enumerate}[nolistsep]
    \item Операции $\min(x_1, x_2)$, $\max(x_1, x_2)$, $x_1 \cdot x_2 \pmod k$, $x_1 + x_2 \pmod k$ ассоциативны и коммутативны.
    \item Дистрибутивности: $(x_1 \vee x_2) \& x_3 = (x_1 \& x_3) \vee (x_2 \& x_3)$, $(x_1 \& x_2) \vee x_3 = (x_1 \vee x_3) \& (x_2 \vee x_3)$, $(x_1 + x_2) \cdot x_3 = x_1 \cdot x_3 + x_2 \cdot x_3$.
    \item При $k > 2$: ${\sim}({\sim}x) = x$, $\overline{\overline{x}} = x + 2 \pmod k$.
    \item ${\sim}(\min(x_1, x_2)) = \max({\sim}x_1, {\sim}x_2)$ --- аналог закона де Моргана. Заметим, что для отрицания Поста аналогичное равенство при $k > 3$ тождеством не является.
\end{enumerate}
