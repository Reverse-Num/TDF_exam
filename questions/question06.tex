\section{Полином Жегалкина. Существование и единственность представления булевой функции в виде полинома Жегалкина.}

\begin{definition}
    \textit{Элементарная конъюнкиця} --- выражение вида $x_{i_1}^{\sigma_1}x_{i_2}^{\sigma_2} \ldots x_{i_k}^{\sigma_k}$, $k\geqslant 1$.
\end{definition}

\begin{definition}
    \textit{Полином Жегалкина} --- выражение вида $\displaystyle \sum_{\{i_1, \ldots, i_s\} \subseteq \{1, \ldots, n\}}a_{i_1\ldots i_s}x_{i_1}\cdot\ldots\cdot x_{i_s}.$
\end{definition}

\begin{remark}
    Можно рассматривать полином Жегалкина, как сумму по модулю 2 произвольного подмножества множества $K^*_n$ конъюнций вида $x_{i_1}x_{i_2} \ldots x_{i_k}$, к которому добавлена константа 1, которую будем называть конъюнкицей длины 0 от пустого множества переменных.
\end{remark}

\begin{theorem}[Жегалкин]
    Каждая функция алгебры логики представима в виде полинома Жегалкина, причём единственным образом.
\end{theorem}

\begin{proof}
    Существование следует из полноты системы $\{0, 1, xy, x \oplus y\}$. Единственность следует из количества полиномов. В самом деле, возможных слагаемых в сумме ровно $2^n = \abs{K^*_n}$. И каждое либо есть в нашей сумме, либо нет. Получаем $2^{2^n} = \abs{P_2(n)}$.
\end{proof}