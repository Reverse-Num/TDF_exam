\section{Порядок роста функции Шеннона в произвольном полном конечном базисе.}

\begin{theorem}
    Пусть $B$ --- конечный полный базис, тогда существуют $a, b > 0$, такие что, при $n\to \infty$ выполняется ассимптотическое неравенство:
    \[
        a\frac{2^n}{n} \lesssim L_B(n) \lesssim b\frac{2^n}{n}
    \].
\end{theorem}

\begin{proof}
    Прямое следствие теоремы 1 из билета 30 и верхней и нижней оценок функции Шенона.
\end{proof}