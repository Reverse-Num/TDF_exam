\documentclass[a4paper, 11pt]{article}
\usepackage{preamble}
\usepackage{import}

\begin{document}

\title{\bfseries\scshape Экзамен по теории дискретных функций}
\date{1 курс$\quad\bullet\quad$Весенний семестр 2024\,г.}
\author{Лектор: Кочергин В.\,В.$\quad\bullet\quad$ Автор: Чепелев Дмитрий\thanks{\texttt{Telegram: @Chepelka\_v\_chepchike. Последняя компиляция: \today}}, группа 105}

\maketitle

\begin{abstract}
    Обо всех ошибках и опечатках пишите мне, исправлю. Хочу выразить особую благодарность Никите Пшеничному, это его оболочка Latex и много хороших вещей я заимствовал из его файла по подготовке к коллоквиуму. Также хочу выразить благодарность за помощь в подготовке файла Егору Скроботову, за помощь в нахождении опечаток Ярославу Додонову.

    Немного о файле: страницы в содержании кликабельны, теоремы носят обязательный характер, предложения, наоборот.
\end{abstract}

\tableofcontents

\newpage

\import{questions/}{question01.tex}
\import{questions/}{question02.tex}
\import{questions/}{question03.tex}
\import{questions/}{question04.tex}
\import{questions/}{question05.tex}
\import{questions/}{question06.tex}
\import{questions/}{question07.tex}
\import{questions/}{question08.tex}
\import{questions/}{question09.tex}
\import{questions/}{question10.tex}
\import{questions/}{question11.tex}
\import{questions/}{question12.tex}
\import{questions/}{question13.tex}
\import{questions/}{question14.tex}
\import{questions/}{question15.tex}
\import{questions/}{question16.tex}
\import{questions/}{question17.tex}
\import{questions/}{question18.tex}
\import{questions/}{question19.tex}
\import{questions/}{question20.tex}
\import{questions/}{question21.tex}
\import{questions/}{question22.tex}
\import{questions/}{question23.tex}
\import{questions/}{question24.tex}
\import{questions/}{question25.tex}
\import{questions/}{question26.tex}
\import{questions/}{question27.tex}
\import{questions/}{question28.tex}
\import{questions/}{question29.tex}
\import{questions/}{question30.tex}
\import{questions/}{question31.tex}
\import{questions/}{question32.tex}
\import{questions/}{question33.tex}
\import{questions/}{question34.tex}
\import{questions/}{question35.tex}
\import{questions/}{question36.tex}
\import{questions/}{question37.tex}
\import{questions/}{question38.tex}
\import{questions/}{question39.tex}
\import{questions/}{question40.tex}
\import{questions/}{question41.tex}
\import{questions/}{question42.tex}
\import{questions/}{question43.tex}


\end{document}
