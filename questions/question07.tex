\section{Замыкание множества функций. Свойства замыкания. Замкнутые классы булевых функций. Классы $T_0$ и $T_1$ функций, сохраняющих константы.}

\begin{definition}
    \textit{Замыканием} $[M]$ множества $M$ функций алгебры логики называется множество всех функций, которые можно получить при помощи операций суперпозиции и введении фиктивных переменных над $M$.
\end{definition}

\begin{definition}
    Множество $M$ называется \textit{замкнутым}, если $[M] = M$.
\end{definition}

\begin{definition}
    Если для множеств $A$ и $F$ булевых функций выполняется равенство $[A] = F$, то говорят, что система $A$ \textit{полна} в $F$.
\end{definition}

\begin{theorem}[Простешие свойства замыкания]
    \begin{enumerate}[nolistsep]
        \item $[M] \supseteq M$ (экстенсивность);
        \item $[[M]] = [M]$ (идемпотентность);
        \item $M_1 \subseteq M_2 \Rightarrow [M_1] \subseteq [M_2]$ (монотонность);
        \item $[M_1 \cup M_2] \supseteq [M_1] \cup [M_2]$;
        \item $[M_1 \cap M_2] \subseteq [M_1] \cap [M_2]$.
    \end{enumerate}
\end{theorem}

\begin{definition}
    $T_0 \vcentcolon = \{f \in P_2: f(0, \ldots, 0) = 0\}$ (<<сохраняют $0$>>), $T_1 \vcentcolon = \{f \in P_2: f(1, \ldots, 1) = 1\}$ (<<сохраняют $1$>>).
\end{definition}

\begin{theorem}
    Классы $T_0$ и $T_1$ замкнуты.
\end{theorem}

\begin{proof}
    Добавление/изъятие фиктивной переменной не выводит за пределы класса.\\ 
    Доказательство проведём индукцией для формулы $\Phi$, которая реализует функцию $g$. База $\Phi = x_i$ --- верно. Предположим, что $\Phi_i$ лежит в $T_0$ как функция $f_i(x_1,\ldots,x_m)$ (при необходимости добавим фиктивные переменные). Рассмотрим произвольную формулу над $T_0$, $\Phi = f(\Phi_1, \ldots, \Phi_n)$, которая реализует функцию
    \[
        g(x_1,\ldots, x_m) = f(f_1(x_1, \ldots, x_m), \ldots, f_n(x_1, \ldots, x_m)).
    \]
    Тогда, поскольку $f, f_i \in T_0$,
    \[
        g(0,\ldots, 0) = f(f_1(0, \ldots, 0), \ldots, f_n(0, \ldots, 0)) = f(0, \ldots, 0) = 0 \Rightarrow g \in T_0.
    \]
    
    А значит, любая суперпозиция функций из $T_0$ также является функцией из $T_0$. Доказательство для $T_1$ аналогично.
\end{proof}

\begin{proposal}
    $\abs{T_0 \cap P_2(n)} = \abs{T_1 \cap P_2(n)} = 2^{2^n - 1}$.
\end{proposal}

\begin{proof}
    Значение для одного набора уже задано, для остальных выбираются так же, как и раньше.
\end{proof}

\begin{proposal}
    $T_0 = [\{xy, x\oplus y\}]$, $T_1 = [\{x\vee y, x\sim y\}]$.
\end{proposal}

\begin{proof}
    Для начала покажем, что $T_0 \subseteq [\{xy, x\oplus y\}]$. Рассмотрим полином Жегалкина для произвольной функции $f \in T_0$. На нулевом наборе, он должен принимать значение 0, а значит, свободный коэффициент в полиноме равен 0. Следовательно, полином получается суперпозициями над $\{xy, x\oplus y\}$. Обратное включение следует из монотонности замыкания: $\{xy, x\oplus y\}\subseteq T_0 \Rightarrow [\{xy, x\oplus y\}] \subseteq [T_0] = T_0$.

    Доказательство для $T_1$ следует из принципа двойственности (см. билет 9).
\end{proof}