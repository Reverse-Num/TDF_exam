\section{Периодические последовательности. Лемма о преобразовании автоматом периодических последовательностей.}

Пусть $\alpha = (\alpha(1), \alpha(2), \ldots)$ --- некоторая последовательность из $A^{\infty}$.

\begin{definition}
    Натуральное число $d$ называется \textit{периодом} последовательности $\alpha$, если существует такой номер $N$, что для любого $t\geqslant N$ выполняется равенство $\alpha(t+d) = \alpha(t)$. Последовательность называется \textit{периодической}, если для неё существует хотя бы один период.
\end{definition}

Поскольку из всех периодов периодической последовательности $\alpha$ можно выбрать минимальный период $d_0$, то все периоды последовательности кратны $d_0$.

\begin{lemma}[Лемма о преобразовании автоматом периодических последовательностей]
    Конечный инициальный автомат с $\lambda$ состояниями преобразует периодическую последовательность с периодом $d$ в периодическую последовательность с периодом $\lambda_1 \cdot d$, где $\lambda_1\in \N, \lambda_1 \leqslant \lambda$.
\end{lemma}

\begin{proof}
    Пусть $V_{q_0} = (A, B, Q, F, G_{q_0})$ --- инициальный автомат, $|Q|=\lambda$, $q_0 \in Q$, $f_{V_{q_0}}(x)$ --- автоматная функция, вычисляемая автоматом $V_{q_0}$, $\alpha = (\alpha(1), \alpha(2), \ldots)$ --- периодическая последовательность $A^\infty$ с периодом $d$, а $\beta = f_{V_{q_0}}(\alpha) = (\beta_1, \beta_2, \ldots)$ --- последовательность из $B^{\infty}$.

    Так как $d$ --- период последовательности $\alpha$, то существует такое $N$, что для любого $t\geqslant N$ выполняется равенство $\alpha(t+d) = \alpha(t)$. Поэтому
    \[
        \alpha(N) = \alpha(N+d) = \alpha(N+2d) = \ldots = \alpha(N+\lambda d).
    \]
    Рассмотрим $\lambda+1$ состояний автомата: $q(N), q(N+d), q(N+2d), \ldots, q(N+\lambda d)$. Так как $|Q| = \lambda$, то среди них найдутся по крайней мере два одинаковых. То есть существуют такие $i, j$, $0\leqslant i < j  \leqslant \lambda$, что $q(N+id) = q(N+jd)$. Положим $\lambda_1 = j-i$. Поскольку состояния в моменты $N+id$, $N+jd$ совпадают и подаваемые последовательности после этих моментов, то эти моменты времени <<неотличимы>>, следовательно выходные символы в эти моменты времени и последующие, отличающиеся на $\lambda_1 d$ совпадают, а значит, $\beta$ --- периодическая с периодом $\lambda_1 d$.
\end{proof}

Обозначим через $A_k$ множество периодических последовательностей из $A^{\infty}$, у которых минимальные периоды не имеют простых делителей больших $k$. Из леммы о периодической последовательности получаем

\begin{corollary}
    Конечный инициальный автомат с $\lambda$ состояниями при $\lambda \leqslant k$ преобразует последовательности из $A_k$ в последовательности из $A_k$.
\end{corollary}