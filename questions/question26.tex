\section{Сложность реализации функции (множества функций) схемами из функциональных элементов. Функция Шеннона. Простые верхняя и нижняя оценки функции Шеннона.}

Если функция $f$ (произвольной природы) реализуется какой-либо схемой $S$ в базисе $B$, то, очевидно, найдутся еще схемы, реализующие функцию $f$ в базисе $B$. 

\begin{definition}
    Определим величину $L_B(f)$ --- \textit{сложность реализации функции} $f$ \textit{схемами в базисе} $B$ --- равенством $L_B(f) = \min L(S)$, где минимум берется по всем схемам $S$, реализующим функцию $f$ в базисе $B$. Схема, в которой достигается минимум называется \textit{минимальной}.
\end{definition}

\begin{definition}
    \textit{Cложность} $L_B(\{f_1, \ldots, f_m\})$ \textit{реализации системы функций} $\{f_1, \ldots, f_m\}$ \textit{схемами в базисе} $B$: $L_B(\{f_1, \ldots, f_m\}) = \min L(S)$, где минимум берется по всем схемам $S$, реализующим систему функций $\{f_1, \ldots, f_m\}$ в базисе $B$.
\end{definition}

\begin{definition}
    \textit{Функцией Шеннона} сложности реализации функций схемами в базисе $B$, будем называть функцию $L_B(n)$, определяемую равенством $\ds L_B(n) = \max_{f\in P_2(n)} L_B(f)$.
\end{definition}

    Договоримся, что в случае, когда набором элементарных операций является классический базис $B_0$, индекс $B_0$ у функционалов сложности будем опускать.

\begin{theorem}[Верхння оценка функции Шенона]
    Для любого натурального $n$ выполняется неравенство
    \[
        L(n) \leqslant n2^n.
    \]
\end{theorem}

\begin{proof}
    Константы можно реализовать схемами сложности 2: $0 = x_1 \cdot \overline{x}_1$, $1 = x_1 \vee \overline{x}_1$. Для любой функции $f(x_1, \ldots, x_n)$ отличных от констант рассмотрим представление в виде СДНФ и последовательно реализуем формулу СДНФ схемой. Тогда потребуется $n$ операций для отрицаний всех переменных, по $n - 1$ операций конъюнкции на каждую из не более чем $2^{n} - 1$ элементарных конъюнкций, а затем не более $2^{n} - 2$ операций дизъюнкции для реализации функции $f$.\\
    Таким образом, 
    \[
        L(n) \leqslant n + (n-1)(2^{n}-1) + 2^n - 2 < n2^n.
    \]
\end{proof}

\begin{theorem}[Нижняя оценка функции Шенона]
    Пусть $B$ --- конечное множество булевых функций, удовлетворяющее условию $[B] = P_2$. Тогда для произвольной булевой функции $f$, существенно зависящей от $n$ переменных, выполняется неравенство
    \[
        L_B(f) \geqslant \ceil{\frac{n-1}{r(B) - 1}},
    \]
    где $r(B)$ --- наибольшее число существенных переменных у функций базиса $B$.
\end{theorem}

\begin{proof}
    Рассмотрим произвольную минимальную схему $S$ в базисе $B$ для функции $f$. Обозначим через $R(S)$ число рёбер в схеме $S$. В силу существенной зависимости функции $f$ от всех $n$ переменных и минимальности схемы $S$ из всех вершин схемы $S$, кроме выходной, выходит по крайней мере по одному ребру, т.\,е.
    \[
        R(S) \geqslant n + L(S) - 1.
    \]
    С другой стороны, в каждый функциональный элемент входит не более $r(B)$ рёбер, поэтому
    \[
        R(S)\leqslant r(B) L(S).
    \]
    Получаем, что
    \[
        n + L(S) - 1 \leqslant r(B)L(S) \quad \Longleftrightarrow \quad L_B(f) = L(S) \geqslant \frac{n-1}{r(B) - 1}.
    \]
\end{proof}