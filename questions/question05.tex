\section{Лемма о сводимости полных систем булевых функций. Существование конечной полной подсистемы в полной системе булевых функций. Примеры полных систем.}

\begin{definition}
    Система $F$ булевых функций называется \textit{полной}, если любая функция алгебры логики выражается формулой над $F$.
\end{definition}

\begin{remark}
    Очевидно, что множество $P_2$ полно. Кроме того, согласно доказанной выше теореме, множество $\{\overline{x}, xy, x \vee y\}$ тоже полно.
\end{remark}

\begin{lemma}[О сводимости полных систем]
    Пусть $F$ --- полная система булевых функций и любая фукнкция из множества $F$ выражается формулой над системой $G$. Тогда $G$ --- полная система.
\end{lemma}

\begin{proof}
    В самом деле, рассматривая формулу для функции $f$ над системой $F$ заменим каждую функцию на её выражение из $G$, получим формулу для произвольной функции $f$ над системой $G$.
\end{proof}

\begin{theorem}
    В любой полной системе булевых функций можно выделить конечную полную подсистему.
\end{theorem}

\begin{proof}
    Пусть есть бесконечная полная система $F$. Раз она полна, то существуют конечные формулы для $\{\overline{x}, xy, x \vee y\}$. Тогда по лемме о сводимости полных систем функции в этих формулах будут образовывать конечную полную подсистему.
\end{proof}

С помощью данной леммы можем привести ещё ряд примеров полных систем:
\begin{enumerate}
    \item $\{\overline{x}, xy\}$ полно, т.\,к. $\{\overline{x}, xy, x \vee y\}$ полно и $x \vee y = \overline{\overline{x} \, \overline{y}}$;
    \item $\{\overline{x}, x \vee y\}$ полно --- аналогично;
    \item $\{x \mid y\}$ полно, т.\,к. $x \mid x = \overline{x}$, а $(x \mid y) \mid (x \mid y) = \overline{x \mid y} = xy$.
    \item $\{x \downarrow y\}$ полно --- аналогично;
    \item $\{0, 1, xy, x \oplus y\}$ полно, т.\,к. $\overline{x} = x \oplus 1$.
\end{enumerate}