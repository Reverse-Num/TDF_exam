\section{Граф (ориентированный и неориентированный). Основные понятия для графа. Геометрическая реализация графа. Изоморфизм графов. Подграф. Подграф, индуцированный множеством вершин. Пути, цепи, циклы на графе. Компоненты связности графа. Связные графы.}

\begin{definition}
    \textit{Графом} $G(V, E, \rho)$, где $V = \{v_1, v_2, \ldots\}$ --- конечное или счётное множество, называемое \textit{вершинами графа}, $E = \{e_1, e_2, \ldots \}$ --- конечное или счётное множество, называемое \textit{рёбрами графа}, а $\rho$ каждому ребру $e$ из множества $E$ сопоставляет элемент из множества $V_2 \cup V^2 \cup V$, где $V_2$ --- множество всех двухэлементных подмножеств множества $V$, $V^2 = V \times V$ --- множество всех упорядоченных пар элементов из $V$.
\end{definition}

Если $\rho(e) = \{v_1, v_2\} \in V_2$, то $e$ называют \textit{неориентированным ребром} графа $G$.

Если $\rho(e) = (v_1, v_2) \in V^2$, то $e$ называют \textit{ориентированным ребром} графа $G$. Говорят, что ребро $e$ выходит из вершины $v_1$ и входит в вершину $v_2$.

В обоих случаях говорят, что вершины $v_1$ и $v_2$ \textit{инцидентны ребру} $e$.

Если $\rho(e) = v$ или $\rho(e) = (v,v)$, то ребро $e$ называется \textit{петлёй} или \textit{ориентированной петлёй} соответственно.


\begin{definition}
    \textit{Неориентированный граф} --- граф, в котором все рёбра неориентированные.
\end{definition}

\begin{definition}
    \textit{Ориентированный граф} --- граф, в котором все рёбра ориентированные.
\end{definition}

Рёбра $e_1, \ldots, e_s$, удовлетворяющие условию $\rho(e_1) = \ldots = \rho(e_s)$, называются \textit{кратными}.

Граф, в котором нет кратных рёбер и петель, называется \textit{простым}. Неориентированный граф, в котором нет кратных рёбер и петель, называется \textit{обыкновенным}.

Обозначим через $\deg v$ число рёбер, инцидентных вершине $v$ (при этом петли считаются дважды). Вершина $v$ называется \textit{изолированной}, если $\deg v = 0$. Вершина $v$ называется \textit{висячей}, если $\deg v = 1$.

\begin{definition}
    Последовательность $v_{s_1}, e_{t_1}, v_{s_2}, e_{t_2}, \ldots, v_{s_k}, e_{t_k}, v_{s_{k+1}}$ называется \textit{путём} от вершины $v_{s_1}$ (\textit{начало пути}) к вершине $v_{s_{k+1}}$ (\textit{конец пути}) длины $k$, $k \geqslant 1$, если для любого $i$, $i = 1, \ldots, k$, либо $\rho(e_{t_i}) = \{v_{s_i}, v_{s_{i+1}} \}$, либо $\rho(e_{t_i}) = (v_{s_i}, v_{s_{i+1}})$.
\end{definition}

\begin{definition}
    Путь, в котором нет повторяющихся вершин, называется \textit{цепью}.
    Путь, в котором нет повторяющихся рёбер и совпадает начало и конец, называется \textit{циклом}.
\end{definition}

\begin{definition}
    Неориентированный граф, в котором любые две вершины соединены путём, называется \textit{связным}.
\end{definition}

\begin{definition}
    Граф $G'(V', E', \rho')$ называется \textit{подграфом} графа $G(V, E, \rho)$, если $V' \subseteq V$, $E' \subseteq E$ и $\forall e \in E'$ верно $\rho'(e) = \rho(e)$. 
\end{definition}

\begin{definition}
    Подграф \(G'(V', E', \rho')\) называется \textit{подграфом, индуцированным множеством вершин} \(V'\) графа \(G(V, E, \rho)\), если \(E'\) состоит из всех рёбер \(e \in E\), для которых \(\rho(e) = \{v_i, v_j\}\) с \(v_i, v_j \in V'\) (в неориентированном случае) или \(\rho(e) = (v_i, v_j)\) с \(v_i, v_j \in V'\) (в ориентированном случае), и при этом \(\rho'(e) = \rho(e)\) для каждого \(e \in E'\).
\end{definition}

\begin{definition}
    \textit{Компонента связности} --- максимальный (по включению) связный подграф, индуцированный каким-то множеством вершин.
\end{definition}

\begin{definition}
    Графы $G(V, E, \rho)$, $G'(V', E', \rho')$ называются \textit{изоморфными}, если существуют биекции $f: V \to V'$, $g: E\to E'$, такие, что $\forall e, v_1, v_2$: если $\rho(e) = \{v_1, v_2\}$, то $\rho'(g(e)) = \{f(v_1), f(v_2)\}$; а если $\rho(e) = (v_1, v_2)$, то $\rho'(g(e)) = (f(v_1), f(v_2))$.
\end{definition}

