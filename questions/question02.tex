\section{Способы задания булевых функций. Булева функция как подмножество вершин $n$-мерного единичного куба}

\noindent \textbf{Способы задания БФ}:
\begin{enumerate}
    \item \textit{Таблица значений}. Выписываем все наборы и значения, которые функция на них принимает.

    Например,\quad
    \begin{tabular}{cc|c|c|c}
        $x$ & $y$ & $x \& y$ & $x \vee y$ & $x \to y$ \\
        \hline
        0 & 0 & 0 & 0 & 1\\
        0 & 1 & 0 & 1 & 1\\
        1 & 0 & 0 & 1 & 0\\
        1 & 1 & 1 & 1 & 1\\
    \end{tabular}
    \item \textit{Булев куб}. Для функции от $n$ переменных рисуем $n$-мерный куб (каждая вершина соответствует набору из $B_n$) и выделяем те, на которых функция принимает значение 1.\\
    Например, функцию $f(x, y, z) = x\oplus y \oplus z$ можно задать так:
    \begin{asy}
        import three;
        import solids;
    
        size(150);
        currentprojection = perspective(6,3,2);
    
        triple A = (0,0,0), B = (1,0,0), C = (1,1,0), D = (0,1,0),
           E = (0,0,1), F = (1,0,1), G = (1,1,1), H = (0,1,1);
    
        revolution b = sphere(B, 0.02);
        revolution d = sphere(D, 0.02);
        revolution e = sphere(E, 0.02);
        revolution g = sphere(G, 0.02);
    
        draw(A--B--C--D--cycle);
        draw(E--F--G--H--cycle);
        draw(A--E);
        draw(B--F);
        draw(C--G);
        draw(D--H);
    
        draw(surface(b), blue,nolight);
        draw(surface(d), blue,nolight);
        draw(surface(e), blue,nolight);
        draw(surface(g), blue,nolight);
    
        label("(000)", A, NW, fontsize(9));
        label("(001)", B, SW, fontsize(9));
        label("(011)", C, SE, fontsize(9));
        label("(010)", D, NE, fontsize(9));
        label("(100)", E, NW, fontsize(9));
        label("(101)", F, W, fontsize(9));
        label("(111)", G, SE, fontsize(9));
        label("(110)", H, NE, fontsize(9));
    \end{asy}
    \item \textit{Описательный способ}. Задаём правила, которые описывают функцию.\\
    Например, $m(x, y, z) = 
    \begin{cases}
        1, & x+y+z \geqslant 2 \\
        0, & x+y+z < 2
    \end{cases}$. \\
    Функцией голосования называется булева функция, принимающая значение 1, когда в наборах значений переменных преобладают единицы, и принимающая значение 0, когда в наборах значений переменных преобладают нули. 
    \item \textit{Формульный способ}. Задаём функцию формулой (см. билет 3). Например, $x\vee y = xy \oplus x \oplus y$.
\end{enumerate}

\begin{exercise}
    Сколько вершин в серединном сечении $n$-мерного куба? Какова асимптотика роста этой функции?
\end{exercise}